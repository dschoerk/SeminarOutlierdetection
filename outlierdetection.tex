
\documentclass[
a4paper,     %% defines the paper size: a4paper (default), a5paper, letterpaper, ...
% landscape,   %% sets the orientation to landscape
% twoside,     %% changes to a two-page-layout (alternatively: oneside)
% twocolumn,   %% changes to a two-column-layout
 headsepline, %% add a horizontal line below the column title
% footsepline, %% add a horizontal line above the page footer
% titlepage,   %% only the titlepage (using titlepage-environment) appears on the first page (alternatively: notitlepage)
 halfparskip,     %% insert an empty line between two paragraphs (alternatively: halfparskip, ...)
% leqno,       %% equation numbers left (instead of right)
 fleqn,       %% equation left-justified (instead of centered)
% tablecaptionabove, %% captions of tables are above the tables (alternatively: tablecaptionbelow)
% draft,       %% produce only a draft version (mark lines that need manual edition and don't show graphics)
% 10pt         %% set default font size to 10 point
% 11pt         %% set default font size to 11 point
12pt         %% set default font size to 12 point
]{scrartcl}  %% article, see KOMA documentation (scrguide.dvi)

\usepackage{todonotes}

\usepackage[english,german]{babel} 
\usepackage[latin1]{inputenc}

\usepackage[T1]{fontenc}
\usepackage{ae,aecompl}

\usepackage{amsmath,amssymb,amstext}
\usepackage{units}
\usepackage{scrpage2}
\usepackage{graphicx}

%%% \mygraphics{}{}{}
%% usage:   \mygraphics{width}{filename_without_extension}{caption}
%% example: \mygraphics{0.7\textwidth}{rolling_grandma}{This is my grandmother on inlinescates}
%% requires: package graphicx
%% provides: including centered pictures/graphics with a boldfaced caption below
%% 
\newcommand{\mygraphics}[3]{
  \begin{center}
    \includegraphics[width=#1, keepaspectratio=true]{#2} \\
    \textbf{#3}
  \end{center}
}

\subject{CVME Seminar}   %% subject which appears above titlehead
\title{Outlier Detection}
\author{Dominik Sch�rkhuber}
\date{Vienna University of Technology, May 2015}

\begin{document}
 \pagenumbering{roman} %% small roman page numbers
 \maketitle
 \tableofcontents
 \listoffigures
 \listoftables
 \pagenumbering{arabic} %% normal page numbers (include it, if roman was used above)

\section{Introduction}
\label{sec:intro}
Outlier Detection is the identification of data which is not conform to an expected pattern. Unlike in clustering, where items of common behavior are grouped together, and outliers may be seen as unwanted noise, we are actively searching with outlier detection methods for anomalies rather than common patterns in data. Outlier detection is a heavily researched topic and has in its generality a wide range of applications. Some examples are fault detection in safety critical systems, supervision of banking systems, video surveillance or intrusion detection in cyber security. In all those cases we want to automatically distinguish usual behavior from abnormal behavior, which may indicate a system breakdown, fraud, etc. Outliers may also be of use in a more positive way. For example human resource management may use outlier detection techniques to find people capable of a very diverse set of skills. From those examples we can conclude that in many cases abnormal data may be of great interest instead of data following a common pattern. 
Until now the very generic term data was used, we now want to closer specify how input data for outlier detection algorithms may look like. A single data instance consists of multiple attributes, similar to descriptors in image analysis. Data attributes can be classified in different types of data like binary, categorial, discrete or continous data. Each data instance may contain one (univariate) or multiple (multivariate) attributes. In the multivariate case the attribute types may be coherent or different among attributes. 
Consider Fig.~\ref{fig:outlierExample} as an abstract example with input data in the attribute space $\mathbb{R}^2$. Each data point consists of two attributes $p=(x,y)$ with $x,y \in \mathbb{R}$. The data is split into several regions. $N_1$ and $N_2$ refer to normal regions, those points describe normal behavior. $O_1$ and $O_2$ are single outliers, $O_3$ is an outlying region. 

\begin{figure}
	\begin{center}
    	\includegraphics[width=0.5\textwidth]{images/outlier_example}
    \end{center}
	\caption{Example for Outlier Detection}
	\label{fig:outlierExample}
\end{figure}

\subsection{Types of Supervision}
\label{sec:types_of_supervision}
Outlier detection methods can be classified by their type of supervision. Besides the actual data instances an outlier detection algorithm may also use additional information for distinction of outliers from inliers. Such information can be class labels as they are used in pattern matching algorithms. The data instances of a training dataset is augumented with class information. This information enables us to generate a predictive model which may classify data instances of test data. Depending on how much a method utilizes this additional information we distinguish three classes of supervision. 
\\
\textit{Supervised methods} make heavy use of labeling information. These techniques require class (labeling) information for each normal and outlier point in the training dataset. Typically a predicitive model is built which enables the algorithm to classify further data points into normals or outliers. The main drawback of supervised methods is it may be very expensive to acquire correctly labeled training datasets. Depending on the field of use a human expert may be required to do a correct labeling or atleast correct a given automatic labeling of training data. 
\\
\textit{Semi-supervised methods} on the other hand only require labeling information for one class of points. Either only normal or abnormal behavior is captured by the training set. It is often very difficult to provide correct labels for both classes of points. For example in a network intrusion detection system it is usually impossible to model all possible attack vectors as outliers in a training dataset. Whereas information of normal use of the network may be automatically acquired. For this reason unsupervised methods with known sets of outliers are not very popular. Since normal behavior is easier to model techniques with known normal datasets are usually used. 
\\
\textit{Unsupervised methods} as a third method does not rely on any additional information. Therefore these methods are widely applicable since no additional information must be provided. Still we need some way of categorising input data points. For example statistic methods can be used to adapt a parametric distribution to normal and/or abnormal data. Based on a statistical test we may distinguish inliers from outliers. Also many techniques rely on the fact that normal data instances are occuring more frequently than abnormal data instances. Therefore frequently occuring patterns are classified as normal, whereas rare patterns are assumed to be abnormal. 

\subsection{Results of Outlier Techniques}
Another important aspect of an outlier detection technique is the manner in which an outlier is reported. In section \ref{sec:intro} we refered to training data as labeled input data. Labeled either as normal data instance or as an outlier. For outputs of outlier detection techniques we distinguish two classes. The first class are labeling outlier techniques. Those techniques put binary labels on each point identifying them as outliers or inliers, just as we did before for training data. Often these techniques are also called hard classifiers \cite{ben2005outlier}. 
On the other hand there are scoring outlier detection techniques, which are also called soft classification techniques. Scoring techniques determine an outlier score which indicates the outlierness of a point. Which is a continuous value ranging e.g. from being an inlier (0) to being an outlier(1). Scoring techniques enable us to create a ranking among data instances. Through this ranking we can find the most outlying point or based on a threshold we can also compute binary labels for results of a scoring approach.

\todo{eventually more basic terms to describe if needed}
global vs local
TypeI TypeII outliers

\section{Density-Based Local Outliers}
For a first method we will look into LOF \cite{breunig2000lof}. This method uses the local densities of data instances to describe its outlierness. Therefore we can classify LOF as a scoring method. This score is called the \textit{local outlier factor}. LOF is local in the sense that it uses the degree of isolation of objects depend on the distances to the local neighbours of a data instance. This local procedure has an import benefit in contrast to other global methods. 

Consider Fig~\ref{fig:global_example} as an example showing be difference between local and global methods. For a simple global method we first compute all distances $d(p,q)$ between data instances. From a fixed point p we check distances to all other data instances. We count the number of distances being greater than $dmin$ a fixed (global) threshold. If this number is greater than $pct$ a certain percentage of points we define this point as an outlier. Applied to the dataset in Fig~\ref{fig:global_example} we need $dmin$ to be big enough such that data instances in $C1$ are not recognised as outliers. Since the distances between points in $C2$ are much smaller all points are correctly identified to be inliers. Also $O1$ is correctly identified as an outlier, but $O1$ is not. This is because the distance value $dmin$ was chosen globally to fit data points in $C1$. See \cite{breunig2000lof} for further details on this example. Using density based local outliers we are able to surpass these shortcomings. 
 
The outlier definition for LOF is based on the k-distance for each data instance. The k-distance is defined to be the distance to the kth-nearest-neighbour. Further LOF defines the reachability distance from p to q to be the maximum of the distance from p to q and the k-distance of p. For each data instance the local reachability density (lrd) is computed by summing up the reachability distances from its k-nearest-neighbours and normalizing it. To compute the final LOF factor for a point p the local reachability densities from the k-neighbourhood are summed and normalized by the lrd of p. 

\begin{equation}
lrd(p) = (\frac{\sum\limits_{o \in N_k(p)}{reachdist(p,o)}}{|N_k(p)|}))^{-1}
\end{equation}
where $N_k(p)$ is the set of k-nearest-neighbors

\begin{equation}
LOF(p) = \frac{\sum\limits_{o \in N_k(p)}{\frac{lrd(o)}{lrd(p)}}}{|N_k(p)|}
\end{equation}

See \cite{breunig2000lof} for equations and further details. The defined LOF equation computes the factor of outlierness. 


\begin{figure}
	\begin{center}
    	\includegraphics[width=0.5\textwidth]{images/global_example}
    \end{center}
	\caption{Global vs. Local Outlier Detection}
	\label{fig:global_example}
\end{figure}


\section{Outlier Detection in High Dimensional Data\cite{aggar2001}} 

\todo{add info from \cite{zimek2012survey}}
Outlier detection has applications in numerous fields, including fraud detection, network analysis and pattern recognition. Many of these fields usually work with high dimensional data. But with high dimensional data also problems arise. Many outlier detections rely on distance metrics (L1-,L2-norm) which become meaningless in higher dimensions. In the last section we looked at LOF, which also suffers from these kind of problems. Since high dimensional data is often used it is meaningful to further investigate those problems. In this section we will describe the upcoming problems for high dimensional datasets.

\subsection{Distance Metrics in High Dimensional Spaces}
An important problem not only in outlier detection but also in other fields like machine learning, statistics and optimization is the meaningfulness of distance metrics in higher dimensions. As a prominent example remember the LOF algorithm which has been previously investigated. According to Aggarwal et. al. \cite{aggarwal2001surprising} the expressiveness of the $L_k$ norm vanishes with a high number of dimensions. More specifically it is shown that if $\lim\limits_{d \rightarrow \infty}{var(\frac{\|X_d\|_k}{E[\|X_d\|_k]})}=0$ then $\frac{Dmax^k_d-Dmin^k_d}{Dmin^k_d} \rightarrow 0$. Which means that the expressiveness gets lower with an increasing number of dimensions. In other words the variance approaches zero, and so does the the distance measure between a maximum and a minimum point in a dataset. Further \cite{aggarwal2001surprising} shows that in a high dimensional space the contrast between $Dmax^k_d-Dmin^k_d$ increases with $d^{1/k-1/2}$. For the manhattan distance this approaches zero, and for the euclidean distance it's constantly 1. For any $k>0$ we approach infinity. For the selection of a metric this means that in higher dimensions $L_k$ norms with lower $k$ are more expressive.

\subsection{Meaningful Dimensions}
\todo{describe more problems from \cite{zimek2012survey}}

\section{Angle Based Outlier Detection \cite{kriegel2008angle}} 
We previously investigated the problems arising with distance metrics in high dimensional spaces. Since these metrics become very inaccurate in high dimensions we would like to avoid them. Angle based outlier detection is one method to successfully avoid flawed distance metrics. Instead of distances angle based methods use the variance of angles relative to a reference point. For the center of a cluster the variance between angles is big, because neighbours appear in all directions. Approaching the border of a cluster the variance of angles becomes smaller. 

\section{Evolutionary Outlier Detection \cite{aggar2001}}

\section{Outlier Detection with Ensembles \cite{nguyen2010mining}} ensembles



image processing aggarwal p19


 \appendix  %% include it, if something (bibliography, index, ...) follows below
 \bibliographystyle{plain}

\cite{aggar2001}
\cite{angiulli2002fast}
\cite{ben2005outlier}
\cite{breunig2000lof}
\cite{chandola2007outlier}
\cite{zimek2012survey}
\cite{aggarwal2001surprising}
\cite{nguyen2010mining}

\bibliography{literature}

\end{document}

